\documentclass[a4paper]{exam}

\usepackage{amsmath}
\usepackage{amssymb}
\usepackage{amsthm}
\usepackage{array}
\usepackage{geometry}
\usepackage{hyperref}
\usepackage{titling}

\newcolumntype{C}{>{$}c<{$}} % math-mode version of "c" column type

\runningheader{CS/MATH 113}{WC06: Proofs}{\theauthor}
\runningheadrule
\runningfootrule
\runningfooter{}{Page \thepage\ of \numpages}{}

\printanswers

\title{Weekly Challenge 06: Proofs\\CS/MATH 113 Discrete Mathematics}
\author{team-name}  % <== for grading, replace with your team name, e.g. q1-team-420
\date{Habib University | Spring 2023}

\qformat{{\large\bf \thequestion. \thequestiontitle}\hfill}
\boxedpoints

\begin{document}
\maketitle

\begin{questions}

  \titledquestion{Perfect Universe}[5] A \textit{perfect universe} is a universe where the combination of any two elements of the universe yields a unique element of the universe (we write the `combination' of element $a$ with element $b$ as the element $ab$), such that the following holds in the universe (henceforth referred to as $U$): 
  \begin{itemize}
  \item Associativity: For all elements $a, b, c$ in $U$, $(ab)c = a(bc)$. 
  \item Existence of an \textit{identity} element: There exists an element $e$ in $U$ such that when combined with any element $a$ in $U$, it does not change $a$ i.e. $\exists e \in U \ni \forall a \in U, ea=ae=a$. If $e$ is such an element of $U$ we call $e$ the \textit{identity} of $U$.
  \item Existence of \textit{enemies}: For each element $a$ in $U$, there exists an element $b$ in $U$ such that $a$ combined with $b$ produces the identity element $e$ of $U$ i.e. $\forall a \in U, \exists b \in U \ni ab= ba = e$. If $b$ is such an element for $a$, we call $b$ the \textit{enemy} of $a$.
  \end{itemize}
  Note that a perfect universe need not be commutative, i.e., it is not necessary for all elements $a, b\in U$ to have the property that $ab=ba$.\\
  
  \begin{parts}
  \item Prove that \textit{in a perfect universe, there is only one identity element}.
    \begin{solution}
      % Enter your solution here.
    \end{solution}
  \item Prove that \textit{in a perfect universe, every element has a unique enemy}.
    \begin{solution}
      % Enter your solution here.
    \end{solution}
  \item Prove that \textit{for any elements $a$ and $b$ of a perfect universe, the enemy of $ab$ is the same as the enemy of $b$ combined with the enemy of $a$.}
    \begin{solution}
      % Enter your solution here.
    \end{solution}
  \end{parts}

\end{questions}

\end{document}